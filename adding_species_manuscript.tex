%%%%%%%%%%%% packages %%%%%%%%%%%%%%%%%%%%%%%%%%
\documentclass[hidelinks]{article}
\usepackage[letterpaper,margin=1.0in]{geometry}
\usepackage[utf8]{inputenc}
\pagenumbering{arabic}
\usepackage{authblk}
\usepackage{graphicx}
\usepackage[singlelinecheck=false]{caption} % singlelinecheck makes single line caption left aligned instead of centered
\usepackage{subcaption}
\usepackage{amsmath}
\usepackage[round]{natbib}
\usepackage{fancyhdr}
\usepackage{longtable}
\usepackage{booktabs}

\pagestyle{fancy}
\fancyhead[R]{\textbf{On population genomic simulations of new model and non-model species: a guide}}
% for figures
\usepackage{graphicx}
\graphicspath{ {Figures/} }


% for highlighting text
\usepackage{xcolor}
\usepackage{soul}

% bibliography
\usepackage[round]{natbib}   % omit 'round' option if you prefer square brackets
\bibliographystyle{plainnat}

% hyperlinks
\usepackage{hyperref}



%commands to format figure and table references in the supplement
\newcommand{\beginsupplement}{%
        \fancyhead[L]{Supplemental Material}
        \setcounter{table}{0}
        \renewcommand{\thetable}{S\arabic{table}}%
        \setcounter{figure}{0}
        \renewcommand{\thefigure}{S\arabic{figure}}%
     }
\newcommand{\stopsupplement}{%
        \setcounter{table}{0}
        \renewcommand{\thetable}{\arabic{table}}%
        \setcounter{figure}{0}
        \renewcommand{\thefigure}{\arabic{figure}}%
     }


% Avoid pandoc bug when there are lists in the body.
\providecommand{\tightlist}{%
\setlength{\itemsep}{0pt}\setlength{\parskip}{0pt}}

\title{Expanding the stdpopsim species catalogue, and lessons learned from realistic genome simulations}

\author[1,*]{M. Elise Lauterbur}
\author[2,*]{Maria Izabel A. Cavassim}
\author[3,*]{Ariella L. Gladstein}
\author[4,*]{Graham Gower}
\author[5,*]{Georgia Tsambos}
\author[6,7]{Jeff Adrion} % moved from above at request
\author[8]{Arjun Biddanda}
\author[6]{Saurabh Belsare}
% \author[6]{Victoria Caudill}
\author[9]{Jean Cury}
\author[10]{Ignacio Echevarria}
\author[11,11]{Ahmed Hasan}
\author[13,13]{Xin Huang}
\author[15]{Leonardo Nicola Martin Iasi} % lastname Iasi
\author[16]{Jana Obšteter}
\author[17]{Vitor Antonio Corrêa Pavinato} % lastname Pavinato
\author[18,19]{David Peede}
\author[20]{Ekaterina Noskova}
\author[21,22]{Alice Pearson}
\author[23]{Manolo Perez}
\author[6]{Chris Smith}
\author[24]{Jeff Spence}
\author[6]{Anastasia Teterina}
\author[6]{Silas Tittes}
\author[25]{Per Unneberg}
\author[26]{Juan Manuel Vasquez}
\author[27]{Ryan Waples}
\author[28]{Anthony Wilder Wohns}
\author[29]{Yan Wong}
\author[30]{Reed Cartwright}
\author[31]{Aaron Ragsdale}
\author[32]{Franz Baumdicker}
\author[33]{Gregor Gorjanc}
\author[34]{Ryan Gutenkunst}
\author[29]{Jerome Kelleher}
\author[6]{Andy Kern}
\author[6,35]{Peter Ralph}
\author[36]{Dan Schrider}
\author[37]{Ilan Gronau}


\affil[*]{\small{These authors contributed equally to the paper.}}
\affil[1]{\small{Department of Ecology and Evolutionary Biology, University of Arizona, Tucson AZ 85719}}
\affil[2]{\small{Department of Ecology and Evolutionary Biology University of California, Los Angeles}}
\affil[3]{\small{Embark Veterinary, Inc., Boston, MA 02111, USA}}
\affil[4]{\small{Section for Molecular Ecology and Evolution, Globe Institute, University of Copenhagen, Denmark}}
\affil[5]{\small{School of Mathematics and Statistics, University of Melbourne, Australia}}
\affil[6]{\small{Institute of Ecology and Evolution, University of Oregon, Eugene OR 97402}}
\affil[7]{\small{AncestryDNA, San Francisco, CA, 94107, USA}}
\affil[8]{\small{54Gene, Inc., Washington, DC 20005, USA}}
\affil[9]{\small{Université Paris-Saclay, CNRS, INRIA, Laboratoire Interdisciplinaire des Sciences du Numérique, UMR 9015 Orsay, France}}
\affil[10]{\small{School of Life Sciences, University of Glasgow}}
\affil[11]{\small{Department of Cell and Systems Biology, University of Toronto, Toronto ON}}
\affil[12]{\small{Department of Biology, University of Toronto Mississauga, Mississauga ON}}
\affil[13]{\small{Department of Evolutionary Anthropology, University of Vienna, Djerassiplatz 1, 1030 Vienna, Austria}}
\affil[14]{\small{Human Evolution and Archaeological Sciences (HEAS), University of Vienna, Austria}}
\affil[15]{\small{Department of Evloutionary Genetics, Max Planck Institute for Evolutionary Anthropology, Leipzig, Germany}}
\affil[16]{\small{Agricultural Institute of Slovenia, Department of Animal Science, Hacquetova ulica 17, Ljubljana, Slovenia}}
\affil[17]{\small{Entomology Dept., CFAES, The Ohio State University, Wooster, Ohio}}
\affil[18]{\small{Department of Ecology and Evolutionary Biology, Brown University, Providence, RI, USA}}
\affil[19]{\small{Center for Computational Molecular Biology, Brown University, Providence, RI, USA}}
\affil[20]{\small{Computer Technologies Laboratory, ITMO University, St Petersburg, Russia}}
\affil[21]{\small{Department of Genetics, University of Cambridge, UK}}
\affil[22]{\small{Department of Zoology, University of Cambridge, UK}}
\affil[23]{\small{Department of Genetics and Evolution, Federal University of Sao Carlos, Sao Carlos 13565905, Brazil}}
\affil[24]{\small{Department of Genetics, Stanford University School of Medicine, Stanford, CA, 94305}}
\affil[25]{\small{Department of Cell and Molecular Biology, National Bioinformatics Infrastructure Sweden, Science for Life Laboratory, Uppsala University,  Husargatan 3, SE-752 37 Uppsala, Sweden}}
\affil[26]{\small{Department of Integrative Biology, University of California, Berkeley, Berkeley, CA, USA}}
\affil[27]{\small{Department of Biostatistics, University of Washington}}
\affil[28]{\small{Broad Institute of MIT and Harvard, Cambridge, MA 02142, USA}}
\affil[29]{\small{Big Data Institute, Li Ka Shing Centre for Health Information and Discovery, University of Oxford, OX3 7LF, UK}}
\affil[30]{\small{School of Life Sciences and The Biodesign Institute, Arizona State University, Tempe, AZ USA}}
\affil[31]{\small{Integrative Biology, University of Wisconsin-Madison, Madison, Wisconsin}}
\affil[32]{\small{Cluster of Excellence - Controlling Microbes to Fight Infections, Eberhard Karls Universität Tübingen, Tübingen, Baden-Württemberg, Germany}}
\affil[33]{\small{The Roslin Institute and Royal (Dick) School of Veterinary Studies, University of Edinburgh, Edinburgh EH25 9RG, UK}}
\affil[34]{\small{Molcular and Cellular Biology, University of Arizona, Tucson, Arizona 85721}}
\affil[35]{\small{Department of Mathematics, University of Oregon, Eugene OR 97402}}
\affil[36]{\small{Department of Genetics, University of North Carolina at Chapel Hill, Chapel Hill, North Carolina 27599}}
\affil[37]{\small{Efi Arazi School of Computer Science, Reichman University, Herzliya, Israel}}

\date{\small{\today{}}}

\begin{document}

\maketitle


\section*{Abstract}

Simulation is a key tool in population genetics, useful for both
methods development and empirical research. An avalance of population
genomic data is in progress, as data is being generated faster than ever before with
efforts such the Earth Biogenome and its affiliated project networks.
This data, coupled with advances in estimating detailed population genetic models, has made acute the need
for more detailed simulations of a wide range of scenarios and model and non-model species.
Many empirical researchers employing population genetics wish to simulate
their study species, but do not know what information is required in order
to do so sufficiently realistically, what is sufficient realism to address
their question, where to find this information, or how to best share it for ease of reproducibility.
In this paper we discuss the elements of a population genomic
simulation model, including the required input data to make
the model a realistic characterisation of a particular species. We also discuss
common pitfalls and major considerations in choosing this input data.
Further, we describe how new species models can be integrated into the
catalog of stdpopsim, a recently-developed tool that makes it easy to simulate
complex population genetic models using up to date information.
So far stdpopsim has been limited to well characterised model species 
such as humans, chimpanzees, and \emph{Arabidopsis},
and we illustrate the process of adding a species to stdpopsim using
examples and lessons learned from a recent hackathon designed to expand the range of supported species.
Thus, this paper provides a means to expand the accessibility of population
genetic simulations to the broader empirical and development population
genetic community by serving as a tutorial in both how to
assemble the data that is required to simulate
a species, and how this information can be incorporated
into the stdpopsim catalog to make it available and accessible to everyone.


\hypertarget{introduction}{%
\section*{Introduction}\label{introduction}}

Dramatic reductions in sequencing costs are enabling the generation of
unprecedented amounts and diversity of genomic data for a huge variety of species
\citep{Ellegren2014}. Ongoing efforts to systematically sequence life on
Earth by efforts such as the Earth Biogenome \citep{Lewin2022} and its
affiliated project networks (for example, Vertebrate Genomes
\citep{Rhie2021}, 10,000 Plants \citep{Cheng2018} and others, see
https://www.earthbiogenome.org/affiliated-project-networks) are
providing the backbone to facilitate enormous increases in population-level genomic data for
new model and non-model species. Methods for inferring
demographic history and natural selection from such data are also flourishing
\citep{Beichman2018}. Past methods development has justifiably focused on the
human genome, or a few key species such as \emph{Drosophila}, 
as model systems. More recently attention is being paid to
generalize methods to include important population dynamics not present
in these models, such as inbreeding \citep{Blischak2020}, skewed offspring
distributions \citep{Montano2016}, selfing (eg. as implemented in Demes
\citep{Gower2022}), intense artificial selection \citep{MacLeod2013,
MacLeod2014}, and to realistically model individual non-model species of interest.

Simulations from population genomic models are an important component
for analysing this new data and testing these models; simulations are thus vital to both methods
development and empirical research. For example, they provide training
data for inference methods based on machine learning \citep{Schrider2018} or
Approximate Bayesian Computation \citep{Csillery2010}. They can also serve as
baselines for further analyses: for example, models incorporating
demographic history serve as null models in selection analyses
\citep{Hsieh2016a} or to seed downstream breeding program simulations
\citep{Gaynor2020}. More recently, population genomic simulations have begun
to be used to help guide conservation decisions for threatened species
\citep{Teixeira2021}.

In general, the usefulness of population genomic simulations increases
the more realistically they represent the species being simulated---that
is, as they incorporate more relevant elements of the species' biology. Important
elements include genome features such as mutation and recombination
rates that strongly affect genetic variation and haplotype structure
\citep{Nachman2002}, particularly when linked selection is important \citep{Cutter2013}. 
Furthermore, the demographic history of a
species, encompassing population sizes, divergences, and gene flow, can
dramatically affect patterns of genetic variation \citep{Teshima2006}. Thus
estimates of these and other evolutionary parameters (e.g. those governing
the process of natural selection) are fundamentally important to the
development of simulations of species of interest. This presents
challenges not only in the coding of the simulations themselves, but in
the choice of parameter estimates to be used to shape the simulation
model.

stdpopsim is a community resource recently developed to provide easy
access to detailed population genomic simulations \citep{Adrion2020}. This
resource lowers the technical barriers to performing these simulations
and reduces the possibility of erroneous implementation of simulations
for species with published models. But so far stdpopsim has been
primarily restricted to well-characterized model species. Feedback from
stdpopsim workshops has emphasized the community desire to simulate
non-model species of interest and ideally incorporate these into the stdpopsim catalog,
the need for a better understanding among the empirical population
genomic community of when it is practical to create a realistic
simulation of a species of interest, and which genomic elements are
necessary and how to choose relevant parameter estimates for them.

The choice of whether and how to develop population genomic
simulations for a species of interest is affected by the intended
analysis and the genomic resources and knowledge available for the
species. These choices have a major impact on the resulting patterns of
genomic variation generated by the simulation. The fundamental
importance of these components of realistic population genomic
simulations is not always well understood, and the necessary choices can
be challenging. While stdpopsim provides a framework for standardizing
simulations of some species, the broader population genetics community
can benefit from additional guidance in making and implementing these
choices to simulate a species of interest.

This paper is intended as a resource for both methods
developers and empirical researchers to develop simulations of their own
species of interest, with the potential to submit the simulation
framework for inclusion in the stdpopsim catalog for peer review
(DRS: is peer review the right phrase for this? Maybe ``independent verification?'') and
community use. In the \textbf{Tutorial}, we discuss the elements of a
population genomic simulation model that realistically characterizes a
species, including when a whole-genome simulation is more useful than
simulations based on either individual loci or generic loci, the required input data (genome
assembly, mutation and recombination rates, and demographic model) and
its quality, common pitfalls in choosing appropriate parameters, and
considerations for how to approach species that are missing some
necessary inputs. This paper is not intended as a tutorial for
implementing simulations in any particular simulator, rather to provide
guidance for what information is sufficient for a realistic simulation
using any simulator, and how to add this information to stdpopsim. The latter 
is discussed in the \textbf{Application} section, where we show how
species models may be integrated into the stdpopsim catalog. This includes
briefly presenting the current method for adding species, clarifying the
required genomic resources, and describing the quality control process
that reflects the peer review of a species model.
(DRS: I am not in love with the changes I have suggested here,
but I felt like in the current version of the text, the Application
part seemed like an afterthought when really it is the crux of the paper
and the titular section, given that this is the ``adding-species-manuscript''!)


\hypertarget{sec2}{%
	\section*{Standardized genome-wide simulations}\label{sec:std-sim}}



% ILAN: the sentence below looks like it would fit well in the intro,
%   so I'm commenting it out here but will later consider moving it
%
%Simulation is one of the key tools in population genetics, but can
%present unexpected challenges and has many hidden pitfalls for the
%unwary population geneticist. 

\colorbox{yellow}{[TODO:  review this frist paragraph, which was rewritten to
	provide a fundamental overview of popgen simulations}\\
\colorbox{yellow}{ (Elise/Peter?)]}
	
	
The main objective of population genomic simulations is to capture as accurately as possible the patterns of sequence variation along the genome within a given population of interest. Most simulations do not assume a specific reference genome sequence, but that does not prevent them from realistically portraying patterns of variation. Modern simulation engines, such as \texttt{msprime} \citep{Kelleher2016,Nelson2020} and \texttt{SLiM} \citep{Haller2019}, are capable of producing realistic patterns of sequence variation if they are provided with accurate estimates for parameters describing the genome architecture and evolutionary history of the simulated species. These parameters describe numerous features, including the chromosome lengths,  mutation and recombination rates, the demographic history of the simulated population, and the landscape of natural selection along the genome. The growing availability of population genomic data has made it increasingly possible to apply computational methods to real genomic data in order to obtain decent estimates for the key parameters required for realistic simulations. Thus, the main challenge when setting up a population genomic simulation is to survey the literature for relevant studies that estimated these parameters, and then correctly plugging these parameter values into the appropriate simulation engine. This step often involves integrations between different literature sources and non-trivial conversions between different scales. As a result, while the simulations themselves may require considerable computation resources, the most time-consuming and error-prone part in population gemomic simulations is setting them up.


The main objective of \texttt{stdpopsim} is to standardize this process as much as possible.
%, focusing on the first two steps of curating the relevant literature and applying 
% parameters correctly to the simulation software.
This standardization has several key advantages \citep{Adrion2020}. First, it makes it easier to compare between different
simulations generated for the same species. Another major advantage of the
\texttt{stdpopsim} framework is the quality control process.
When a contributing researcher wishes to add a new simulation model to the catalog,
they flag it for review. Then, another contributor independently tries to create a simulation model based on the same literature sources and the documentation provided by the initial contributor. The two separate models are then
compared to each other by automated scripts. If discrepancies are found, they
are resolved between the two contributors, and if necessary, input of additional members of the community is solicited. This process quite often
finds subtle bugs  \citep{Ragsdale2020} or highlights parts of the model that are
ambiguously defined by the literature sources. 
Therefore, this quality control considerably increases the reliability of the
resulting simulations in any downstream analysis.



One of the things we advocate in \texttt{stdpopsim} is simulation of complete
chromosomes. Such simulations are typically more complex because they require
setting many parameters pertaining to genome structure, and they require
considerable computational resources. Because of this, most simulations carried
out for individual studies avoid simulating complete chromosomes and resort to
separate simulations of many short segments. While this results in simpler
simulations that can be executed concurrently, % and thus much more efficiently, 
simulations of short segments ignore the structure of physical linkage along
chromosome. This linkage affects patterns of variation in subtle ways, which could
have significant influence on different aspects of downstream population genetic
analysis \citep{Nelson2020}. 


\noindent\colorbox{yellow}{[ TODO: check that the main arguments are clearly covered below (Andy/Peter?)]}

First, the correlation between different sites along a chromosome reduces
variation. This has a particularly striking effect in long stretches of
low recombination rates, as observed on the long arm of human chromosome 22 \citep{Dawson2002}.
Second, the lengths of shared haplotypes within and
between populations provides valuable information about recent demographic history.
Methods that use such information, such as MSMC \citep{Schiffels2020}, require as
input very long genomic segments, with realistic recombination rates.
Thus, even when conducting simulations under neutral evolution, short segments
would not be able to capture important features observed in neutral regions of
real genome sequences, such as long-range linkage disequilibrium.
%
When conducting simulations with natural selection, linkage has
an even stronger effect. Selection acting on a small number of sites drives down
genetic variation in nearby sites. This has been argued to be one of the causes
for the observed correlation between recombination rate and genetic diversity
\citep{Begun1992}. Selection at linked sites has been shown to have a widespread
effect on patterns of genome variation in different species \citep{McVicker2009,Charlesworth2012}. This effect tends to be stronger in species
with smaller effective population sizes, but recent selective sweeps can dramatically
reduce variation even in species with large population sizes \citep{Lynd2010}.
%
% ILAN: commented out this part on genetic load. We can mention this briefly,
%   if we think it's important
%
%Third, linkage has the potential to affect demographic realism. The
%genetic load in a simulation of a small segment of chromosome with
%deleterious mutations will necessarily be less than that in whole
%chromosome. This situation makes it easy to simulate unrealistically
%high levels of load without realizing it, if only small unlinked segments are used.
%(TODO: What is the direct effect on demographic realism of the effect of
%load?) (DRS: I think that the three examples given in this section
%aren't the most compelling. The point of the chr22 example isn't super clear,
%the chromosome ends thing seems like a minor issue especially given that
%rec rates there are generally pretty low, and the load case seems to be
%more an issue of ``small scale sims let you simulate unrealistic secnarios''
%rather than ``small scale sims prevent you from simulating realistic scenarios''
%

\noindent\colorbox{yellow}{[ TODO: elaborate on example of Hsieh, et al. below (Ryan?) ]}

Finally, the whole genome sequence can provide important comparative
data with which to evaluate the simulations. This is a powerful tool for
making sure the simulations correspond to the elements of biological
realism that are important for the intended analyses \citep{Hsieh2016a}. 
%
That said, for some purposes it might be completely fine to simulate short sequence
segments or even independent (unlinked) sites. However, this should be done after
careful consideration  of the benefits of whole-chromosome simulations.


\hypertarget{sec3}{%
	\section*{Updates to \texttt{stdpopsim} and the species catalog}\label{sec:expanded-catalog}}

Since its initial publication in \cite{Adrion2020}, several updates have been made to  the simulation framework of \texttt{stdpopsim} and to its species catalog. One fundamental update was to upgrading the neutral simulation engine, \texttt{msprime}, from version 0.7.4 to version 1.0 \citep{Baumdicker2022}. This introduced a few important features, such as %
\colorbox{yellow}{[ TODO: summarize key updates in msprime v1.0 in 1-2 sentences (Franz/Jerome?) ]}.
%
Another key feature added to the simulation framework of \texttt{stdpopsim} was modeling recombination by gene conversion. %
\colorbox{yellow}{[ TODO: summarize model of GC in stdpopsim in 5-7 sentences (Franz?) ]}
%
The two aforementioned updates apply to neutral simulations. We also made significant updates to enable realistic simulations of natural selection using the \texttt{SLiM} engine \citep{Haller2019}. %

\colorbox{yellow}{[ TODO: add 5-7 sentences on uploading annotations, DFEs and mention selection manuscript (Peter/Andy?) ]}
%



\begin{figure}
	\includegraphics[width=\linewidth]{./figs/species_fig.png}
	\caption{Phylogenetic tree of species available in the \texttt{stdpopsim} catalog. 
		In blue are species we published in the original release \citep{Adrion2020}, and in orange are species added by members of the community during the past two years. 
		\colorbox{yellow}{[ TODO: update fig using updated catalog and mark in diff color species added during the hackathon (Elise?) ]}}
	\label{fig:tree}
\end{figure}


\colorbox{yellow}{[TODO: fill in the three missing numbers marked by X in this paragraph (Elise?) ]}

In addition to expanding the simulation capabilities of \texttt{stdpopsim}, a parallel effort has been made to expand the species catalog. When first
published, the \texttt{stdpopsim} catalog included six species: \emph{Homo sapiens}, \emph{Pongo abelii}, \emph{Canis familiaris}, \emph{Drosophila melanogaster}, \emph{Arabidopsis thaliana}, and \emph{Escherichia coli} (Figure \ref{fig:tree}).
One dimension of expansion was introducing additional demographic models for \emph{Homo sapiens}, \emph{Pongo abelii}, \emph{Drosophila melanogaster}, and \emph{Arabidopsis thaliana}. This enables more realistic simulations for these model species. 
However, the true potential of \texttt{stdpopsim} is in allowing easy access to simulations for non-model species. To that end, the PopSim consortium has made a concerted effort to recruit members of the population and evolutionary genetics community to add new species to the stdpopsim catalog. The peak of this effort was a ``Growing the Zoo'' hackathon, that we organized alongside the 2021 ProbGen conference. To prepare people to the hackathon, we organized a series of \colorbox{yellow}{X} introductory workshops to \texttt{stdpopsim} in the four months leading to the hackathon. These workshops allowed us to reach out to a broad community of more than 150 researchers, many of which expressed interest in adding non-model species to \texttt{stdpopsim}. The hackathon was then structured based on feedbacks from these participants. One month before the hackathon, we organized a final workshop to prepare interested participants for the hackathon, by introducing them to  the process of developing a new species model and adding it to the \texttt{stdpopsim}
code base. 
%All members of the population genetic community who were 
%familiar with the this process, either from attending the
%``Growing the Zoo'' workshop or their own previous work, were invited to
%the hackathon.
Roughly \colorbox{yellow}{X} scientists participated in the hackathon, which resulted in 12 species added to the \texttt{stdpopsim} catalog.
This initial concentrated effort later resulted in the addition of \colorbox{yellow}{X} more species during the year following the hackathon (Figure \ref{fig:tree}).
% and three additional species have been added by
%community members outside of the hackathon. (Figure \ref{fig:tree}),
%and started the ball rolling on expanding the
%stdpopsim catalog further.
The concentrated effort to add species to the \texttt{stdpopsim} catalog has lead to a series of important insights about this process, which are summarized in the following section.





\hypertarget{sec4}{%
	\section*{Guidelines for implementing a population genomic simulation}\label{sec:sim-guidelines}}

\colorbox{yellow}{[TODO: careful review of this section. Quite extensive edits since last version (Elise/Dan S/...?) ]}\vspace{1em}
 

Implementing a realistic population genomic simulation requires setting many parameters by integrating information from several publications. In this section, we outline the different pieces of information that need to be gathered, and we provide guidelines in how to use them to set the simulation parameters. 

\subsection*{Basic setup for chromosome-level simulations}

We start by describing the five fundamental categories of model parameters that one has to specify for any (neutral) chromosome-level simulation.


\begin{enumerate}
\def\labelenumi{\arabic{enumi}.}
\item
  \textbf{A chromosome-level genome assembly}, which consists of a list of chromosomes or scaffolds and their lengths. Having a good quality assembly with complete chromosomes, or at least very long scaffolds, is the cornerstone of chromosome-level population genomic simulations (see discussion in \textbf{Standardized genome-wide simulations}). Currently, the number of species with complete chromosome-level assemblies is somewhat limited, but we expect this number to dramatically increase in the near future due to genome initiatives such as the Earth Biogenome \citep{Lewin2022} and its affiliated project networks (e.g.,
  Vertebrate Genomes \citep{Rhie2021}, 10,000 Plants \citep{Cheng2018}).
  % see https://www.earthbiogenome.org/affiliated-project-networks).
  Furthermore, the development of new long-read sequencing technologies
  \citep{Amarasinghe2020} and concomitant advances in assembly pipelines
  \citep{Chakraborty2016} are likely to boost these initiatives. When expanding the \texttt{stdpopsim} catalog, we decided to focus on species with genome assemblies with near-complete gnemome assemblies and fewer than \colorbox{yellow}{X} scaffolds 
  \colorbox{yellow}{[TODO: do we want to specify a specific number here? (Jerome/Peter?)]}. This restriction was set mainly because species with less established genome builds typically do not have decent genetic maps or good estimates of recombination rate, making chromosome-level simulation much less useful. Therefore, the utility of adding such species to the catalog does not justify the considerable excess storage burden incurred by the large data files that specify such partial assemblies. 
  %Ilan: IS THIS FINAL STATEMENT CLEAR ENOUGH? I THINK WE SHOULD SAY SOMETHING ABOUT STORAGE CONSIDERATIONS
  

  
  
\item
  \textbf{An average mutation rate} for each chromosome (per generation per site).  
  This rate estimate can be based on sequence data from pedigrees, mutation accumulation studies, or comparative genomic analysis calibrated by fossil data (i.e., phylogenetic estimate). 
  %If none of these are available for your species of interest, you may use an estimate obtained for another, closely related, species.
  
\item
  \textbf{Recombination rates} (per generation per site).  
  Ideally, a population genomic simulation should make use of a chromosome-level \textbf{recombination map}, since recombination rate is known to vary widely across chromosomes and this affects the patterns of linkage disequilibrium. When that is not available, we suggest specifying an average recombination rate for each chromosome.
  At minimum, an average genome-wide recombination rate needs to be specified, and this is typically available for well assembled genomes. 
  %As with the average mutation rate, if an estimate is not available for your species of interest, you may use an estimate obtained for another, closely related, species.
  
\item
  \textbf{A demographic model} describing ancestral population sizes, population split times and migration rates. A given species might have more than one demographic model, depending on the studied populations, the focus of the demographic study (e.g., population growth or migration rates), and the computational methods used to obtain the model from sequence data. Thus, you should select a demographic model that best fits the focus of your specific study. Misspecification of the demographic model can generate unrealistic patterns of genetic variation, that will affect downstream analyses \citep[e.g.,][]{Navascues2009}. If possible, one should use a detailed demographic model with multiple populations, migration between populations, and fine-grained changes in population sizes. At a minimum, simulation requires a single estimate of \textbf{effective population size}. This estimate, which corresponds to the historic average size, should reproduce in simulation the average observed genetic diversity in that species. Note, however, that this average effective population size will not capture features of genetic variation that are caused by significant fluctuations in population size and population structure \citep{MacLeod2013}. For example, a recent population expansion will produce
  an excess of low frequency alleles that no simulation of a constant-sized
  population will reproduce.

\item
  \textbf{An average generation time} assumed for the species. 
  This parameter is an important part of the species natural history.
  Interestingly, however, it does not directly affect the simulation, since 
  simulation engines typically operate in time units of generations. Thus, the average generation time is primarily used to convert time units to years, and is useful when comparing between different demographic models.
  
  \colorbox{yellow}{[TODO: I wasn't sure if stdpopsim strictly required this or not. Anything more to say here? (Jerome/Peter?)]}
  
\end{enumerate}

\colorbox{yellow}{[TODO: see if the paragraph below covers the main points (Andy?)]}

Theses five categories of parameters are sufficient for generating simulations under neutral evolutions. Such simulations are useful for testing demographic models, but they cannot be used to model the influence of natural selection on patterns of genetic variation.
Simulating genomes under realistic models of natural selection has become increasingly important in order to study the underlying biological function of different genomic regions \colorbox{yellow}{[TODO: add citation(s)]}.
% ILAN: I replaced the consideration below with simpler arguments. need to check that nothing is missing.
% However, the goal of much population genetics work is to understand the action or
% consequences of selection. Although analytical models tend to study the
% effect of single loci under selection in isolation, it has been
% demonstrated that the effects of linked selection can vary substantially
% along the genome in many species \citep{Wolf2017}, although even the general
% strength and nature of this selection is still unknown. 
To generate realistic simulations of natural selection, the simulator needs to know the locations of the selected sites along the genome and the nature and strength of selection in these sites. We have recently developed a framework in \texttt{stdpopsim} to define these features for genomes of interest  \colorbox{yellow}{[TODO: add citation for selection paper]}. This framework involves specifying two additional features into the species model, and using a forward-in-time simulation engine that can incorporate them (e.g. \texttt{SLiM} \citep{Haller2019}):

\begin{enumerate}
	\def\labelenumi{\arabic{enumi}.}
	\setcounter{enumi}{5}
	\item
	\textbf{Genome annotations}, specifying the location of selected sites in a GFF3/GFF file. The annotations can contain information on the location of coding regions, the position of specific genes, and conserved non-coding regions. Regions not covered by the annotation file are assumed to be neutrally evolving.
	\item
	\textbf{Distributions of fitness effects} (DFEs) for the different classed of sites in the annotation file. Each DFE describes the relative frequencies of deleterious, neutral, and beneficial mutations. This distribution is important for understanding the rate of adaptive and stabilizing evolution. DFEs can be inferred from population genomic data, and are available for several species \colorbox{yellow}{[TODO: add citations]}.
\end{enumerate}

\subsection*{Extracting model parameters from the literature}
%
%
For simulations to be useful, it is important to set the model parameters specified above based on values estimated using analysis of relevant genomic data sets. Thus, every parameter should ideally be supported by a \emph{citeable publication} that describes the relevant analysis. Indeed, this is a strict requirement in models added to the \texttt{stdpopsim} catalog. The citation attached to each parameter allows potential users of the catalog to assess the relevance of each model to their study. Another key practice promoted by \texttt{stdpopsim} is independent evaluation of species models. Each model added to the catalog is independently recreated or thoroughly reviewed by a separate researcher. This practice often finds subtle bugs in the suggested model and helps increase the reliability of models published in the catalog. We thus highly recommend this practice also for simulations generated outside of \texttt{stdpopsim}.
% Furthermore, all of these parameters should be documented, citeable, and
%ideally chosen after discussion with another researcher. 
%
% comment by DRS on the above sentence: Why should parameters be chosen after discussion
% with another researcher? What kind of researcher? And what should the nature of this
% discussion be? I am not sure I agree with this, really. The goal should be to
% accelerate research by making it easier for researchers to simulate genomes, but if
% they have to consult with colleagues about most of the choices they make when
% designing the simulations maybe that somewhat defeats the purpose?
% Ilan's response: I think the intention was to emphasize the importance of QC.
% I tried to make this point more directly in the last few sentences of the paragraph,
% replacing the sentence about the need to discuss with other researchers.

%\hypertarget{additional-considerations}{%
%\subsubsection*{Additional considerations}\label{additional-considerations}}

Obtaining reliable and citeable estimates for all model parameters is not a trivial task. Oftentimes, different parameters will be extracted from different publications. For example, it is not uncommon to find an estimate of a mutation rate in one paper, a recombination map in a separate paper, and a suitable demographic model in a third paper. This practice is completely fine, but integrating information from different publications requires some care, because some of these parameters are entangled in non-trivial ways.
For instance, the demographic model may have been estimated in the third paper assuming a mutation rate that has since been overwritten by the mutation rate estimate from the second paper. Naively using the demographic model, as published, with the new estimate of mutation rate will lead to levels of genetic diversity that do not fit the genomic data. This may be fixed by scaling all demographic parameter by the appropriate factor.
However, it is not always clear that the original demographic model with the outdated mutation rate is fully equivalent to the scaled demographic model with the updated mutation rate. This may also depend on other assumptions made by the demography inference method, such as assumptions on recombination. Therefore, when possible, the demographic model, mutation rates, and recombination rates should
be drawn from the same study. If one needs to mix different sources, this should be approached with deliberation, and review by other researchers may be vital.


\colorbox{yellow}{[TODO: write short paragraphs on liftovers (Andy?)]}

%\hypertarget{what-if-we-dont-know-everything}{%
%\paragraph{What if we don't know everything?}\label{what-if-we-dont-know-everything}}
\subsection*{Filling out the missing pieces}

For some species it may be difficult to obtain citeable values for the necessary model parameters, even when combining different sources. We provide several suggestions for dealing with this scenario (see Table \ref{tab:param-mod}).
If your species of interest does not have citeable mutation or recombination rates, you may use average genome-wide rates published for a closely related species. This may slightly skew the average genetic diversity and the average genetic linkage, but will still likely produce useful simulations. A similar approach can be applied to the DFE if you wish to simulate natural selection.
For species that lack a detailed demographic model inferred from genetic data, we suggest to use an average effective population size (\emph{Ne}), which best fits the average observed genetic diversity ($\theta$) in that species. There are various simple formulas to estimate $\theta$ from the number of segregating sites in a population \citep{Watterson1975} or the heterozygosity rate \citep{Tajima1989} \colorbox{yellow}{[TODO: is there an earlier/better reference for $\pi$-based estimate of $\theta$? (Peter?)]}. This estimate of $\theta$ can then be converted to an effective population size by the formula: $Ne=\frac {\theta} {2p\mu}$, where $p$ is the ploidy of the species ($p=1$ for haploid species and $p=2$ for diploid species), and $\mu$ is the average mutation rate assumed for the species.

Several researchers who participated in our hackathon in 2020 wished to add simulation models for species with partial genome assemblies. Many species genome assemblies
are composed of many relatively small contigs whose relation to each
other is not fully known. One way to deal with this situation is to include only the  longer contigs or scaffolds, treating them as separate chromosomes in the simulation. Note that we expect some of the contigs to map to the same chromosome, and modeling them separately will not capture the genetic linkage between them. However, this likely provides a reasonable approximation, at least to genomic segments far enough away from the contig edges. The short contigs can either be omitted from simulation, or lumped together into one (or several) longer pseudo-chromosome. Recall that due to storage constraints, we require species added to the \texttt{stdpopsim} catalog to contain at most \colorbox{yellow}{X} chromosomes or scaffolds \colorbox{yellow}{[TODO: fill in missing number; see above]}. Finally, while whole-chromosome simulations are crucial for many applications, for some purposes, such as demography inference, it may be sufficient to rely on simulation  of many unlinked sites \citep{Gutenkunst2009,Excoffier2013}. In those cases, one may generate useful simulations even without a genome assembly.

%The alternative is to leave aside such strict
%matching, instead simulating an anonymous chromosome from which patterns
%of genetic variation can be extracted (if important, in chunks of size
%similar to the contigs). The latter is usually more realistic,
%since this includes linkage between sites that share a chromosome but
%may be on different real contigs. As noted, this can affect patterns of neutral
%genetic diversity and perhaps more crucially, linked selection. Contig-level
%assemblies are sometimes annotated, and simulations of regions around
%specific genes or genomic features (eg. exons) may be of interest.
%However using the precise location of genes on many, unlinked contigs is
%a false precision because of the importance of modeling linked selection
%across contigs. For this reason, the question of ``realism'' is most
%relevant for those species having chromosome-level assemblies. In our
%experience it is uncommon that a species have widely-used demographic
%models but not a chromosome-level assembly.
%
% Comment by DRS on the above text: I kind of got lost in the above paragraph,
% especially this part and the few sentences after: ``instead simulating an
% anonymous chromosome from which patterns of genetic variation can be extracted
% (if important, in chunks of size similar to the contigs)''. In this part,
% are we talking about randomly assembling contigs into chromosomes? Or adding
% sequence onto the ends of contigs to make each of them into larger chormosomes?
% In the former we can have larger-scale linked selection and all that good stuff,
% but if we extract data from each ``contig'' in our simulated chromosome we have
% to be aware that the linkage these contigs experience with each other differs
% from that of the real genome, since our assembly was arbitrary. We don't have to
% worry about this with the latter strategy because we only sample one
% ``contig'' from a simulated chromosome, although we have to think about what to
% put on the rest of that chromosome.) 



% Please add the following required packages to your document preamble:
% \usepackage{booktabs}
\begin{table}[]
	\captionof{table}{Guide to missing parameters. \\
	%	
	\colorbox{yellow}{[TODO: reconsider table. Either clarify considerations or just defer to the text, which covers this quite clearly.}
	\colorbox{yellow}{Current version is too vague.}
} \label{tab:param-mod} 
	\begin{tabular}{p{1.5in}p{2.2in}p{2.2in}}
			\hline
			Missing parameter  & Options & Considerations \\
			\hline
			Mutation rate      & 
			borrow from closest relative with a citeable mutation rate & 
			will affect levels of polymorphism \\                                                                                          						\hline
			Recombination rate & 
			borrow from closest relative with a citeable rate &
			will affect patterns of selection, linkage, and linked selection
			\\
			\hline
			Demographic model & 
			at least Ne is required and is estimable from mutation rate and genetic data     & 
			the demographic history (e.g. bottlenecks, expansions, admixture) affects patterns of variation substantially {[}CITE{]}, a constant Ne is not ideal \\
			\hline
	\end{tabular}
\end{table}


\hypertarget{sec5}{%
	\section*{Examples of added species}\label{sec:std-sim}}

In this section, we provide examples of two species recently added to the \texttt[stdpopsim] catalog, to demonstrate the key considerations made in the process.

\hypertarget{ano-gambea}{%
	\subsection*{\texorpdfstring{\emph{Anopheles gambiae} (mosquito)}{Anopheles gambiae (mosquito)}}\label{bos-taurus}}

\colorbox{yellow}{[TODO: This section was completely rewritten based on examination of the codebase.}\\
\colorbox{yellow}{This should be carefully reviewed (Andy?)]}

Anopheles gambiae, also known as the African malaria mosquito, is a good example of a non-model organism whose population history has direct implications on human health. Several large scale studies in the recent years have provided information about the population history of this species, on which population genomic simulations can be based \citep{Pombi2006,Miles2017} \colorbox{yellow}{[TODO: add more citations? (Andy?)]}. The simulation model for this species is based on the AgamP4 genome assembly \citep{Sharakhova2007}. The lengths of each of the six chromosomes, including a sex chromosome and the mitochondrial genome, were downloaded directly from Ensembl \citep{ensembl2021}.
The \texttt{stdpopsim} repository has several utilities that interact with Ensembl, making it easy to accurately retrieve basic genome information and construct the appropriate python data structures.
%
%A partial list of the
%genomes housed on Ensembl can be found at
%https://metazoa.ensembl.org/species.html. (DRS: this is kinda just floating here.)
%
Estimates of average recombination rates for each of the five chromosomes (excluding the mitochondrial genome) were taken from a recombination map inferred by \cite{Pombi2006}.
%

\colorbox{yellow}{[TODO:  maybe add table similar to the one in the docs]} \href{https://popsim-consortium.github.io/stdpopsim-docs/latest/catalog.html#sec_catalog_AnoGam_genome}{[table link]}

Since we could not find a citeable estimate for mutation rate for Anopheles gambiae, we used a genome-wide average mutation rate of $\mu=3.5e-09$ mutations per generation per site, estimated for Drosophila Melanogaster by \cite{Keightley2009} using mutation accumulation lines. Using this mutation rate and an estimate of $\theta=4\mu Ne$, the effective population size was set to $Ne=1,000,000$.
%
Importantly, the same mutation rate was also assumed when inferring a more detailed demographic history for this species \citep{Miles2017}. The demographic model inferred by \cite{Miles2017} specifies population size changes throughout the past 11,260 generations in 67 time intervals. During this time period, the population size was inferred to have fluctuated from below 80,000 (an ancient bottleneck roughly 10,000 generations ago) to the present-day estimate of over 4 million %
To convert the time scale from generations to years, we suggest using an average generation times of $1/11$, which was also used in \citep{Miles2017}.

\colorbox{yellow}{[TODO:  maybe add figure similar to the one in the docs]} \href{https://popsim-consortium.github.io/stdpopsim-docs/latest/_images/sec_catalog_anogam_models_gabonag1000g_1a17.png}{[fig link]}


All of these parameters were set in the appropriate source files in the \texttt{stdpopsim} catalog, accompanied by the relevant citation infromation.
The species model was underwent the standard quality control process before it was added to the catalog. It may be refined in the future by adding more demographic models or updating the mutation rate estimate or the recombination map. Note that if in the future we obtain a direct estimate of mutation rate for Anopheles gambiae, then the demographic model mentioned above should be appropriately rescaled to be used with the new mutation rate.


%  Find citeable resources describing the required population and species
%  genetic parameters as detailed in \textbf{Implementing a population genomic simulation}.

% LOW LEVEL DETAILS COMMENTED OUT
%  Open a GitHub account, fork the stdpopsim GitHub
%  repository, and start a pull request by following the steps provided
%  in the ``Adding a new species'' section of the Development chapter in
%  the stdpopsim docs, currently at
%  %\url{https://popsim-consortium.github.io/stdpopsim-docs/stable/development.html?highlight=adding\%%20species\%20catalog\#adding-a-new-species}
%  These steps as they stand in April 2022 are described in detail in the
%  supplementary material, but are subject to change as the stdpopsim
%  framework improves.
%\end{itemize}
%
%\texttt{stdpopsim} uses git for version control. You need to start by opening a GitHub 
%account and forking the \texttt{stdpopsim} repository. Our first step will be to 
%create an upstream link to the version of the repository owned by
%\texttt{popsim-consortium}, then we create a new branch of stdpopsim
%using git to keep track of the new species model we are adding.
%
% The next steps flesh out the templates, which include a directory for
% the species (AnoGam, for our \emph{Anopheles gamiae} example), and three
% files. The first of these files is a data dictionary with space for the assembly
% accession number, the assembly name, and the
% chromosome names associated with their lengths. 
%
%\begin{verbatim}
%_recombination_rate = {
%"2L": 0, # setting to zero because of inversion
%"2R": 1.3e-8,
%"3L": 1.6e-8,
%"3R": 1.3e-8,
%"X": 1e-8,
%"Mt": 0
%}
%\end{verbatim}
%
%Every piece of information requires a citable reference. To include that we
%can create a \texttt{stdpopsim.Citation} object in the same
%\texttt{species.py} file. That object looks like this:
%\begin{verbatim}
%_PombiEtAl = stdpopsim.Citation(
%doi="https://doi.org/10.4269/ajtmh.2006.75.901",
%year=2006,
%author="Pombi et al.",
%reasons={stdpopsim.CiteReason.REC_RATE},
%)
%\end{verbatim}



\hypertarget{bos-taurus}{%
	\subsection*{\texorpdfstring{\emph{Bos
				taurus} (cattle)}{Bos taurus (cattle)}}\label{bos-taurus}}

\colorbox{yellow}{[TODO: review this subsection and highlight high-level considerations and QC (Ilan) ]}


\emph{Bos taurus} (cattle) was added to the \texttt{stdpopsim} catalog during the 2020 hackathon because of its agricultural importance. Agricultural species experience
strong selection due to domestication and selective breeding, leading
to a reduction in effective population size. These processes,
as well as admixture and introgression, produce patterns
of genetic variation that can be very different from typical model
species \citep{Larson2013}. In cattle, these processes have occurred over a
relatively short period (\textasciitilde 10,000 years or less) and are
increasingly intensified to improve food production \citep{Gaut2018,
MacLeod2013}. In cattle, high quality genome assemblies are now
available for several breeds \citep[e.g.,][]{Rosen2020, Heaton2021,
Talenti2022} and population genomic analyses have become widely used to
improve selective breeding and genomic prediction \citep{Meuwissen2001,
MacLeod2014, Obsteter2021}. Their small effective population size
(\textasciitilde 90 around 1980, and continuing to decline due to intense
selective breeding \citep{MacLeod2013, VanRaden2020, Makanjouloa2020}) is
challenging demographic and selection inference \citep{MacLeod2013,
Hartfield2022} and genome-wide association and prediction
\citep{MacLeod2014}. For these reasons, it was useful to develop a
cattle model for stdpopsim.

With respect to the parameters chosen in the stdpopsim implementation,
for the basic genome simulation, we used the most recent assembly
\citep{Rosen2020}, a mutation rate of \(1.2 \times 10^{-8}\) \citep{Harland2017},
a recombination rate of \(0.926 \times 10^{-8}\) \citep{Ma2015}, an
effective population of 62,000 \citep{MacLeod2013}, and a generation interval of 5
years \citep{MacLeod2013}. The chosen effective population size requires
an explanation. Domestication and selective breeding have significantly reduced
effective population size in cattle - from 62,000 at 33,154 generations ago in
ancient cattle to 90 around 1980 in the Holstein dairy breed \citep{MacLeod2013}.
Simulations with the current effective population size of 90 will generate very
low levels of overall genetic diversity, while simulations with the ancient
effective population size of 62,000 will not generate realistic genome structure
in line with the observed genomes \citep[e.g.,][]{Rosen2020}. We have set
the species effective population to 62,000 and strongly suggest complementing the
basic genome simulation with a demographic model. We implemented
the \cite{MacLeod2013} demographic model of the Holstein breed, which was
inferred from runs of homozygosity in the whole-genome sequence of two
iconic bulls. \cite{MacLeod2013} assumed
recombination and mutation rates of \(10^{-8}\) in inferring the
demographic model, but revised the mutation rate to \(0.94 \times 10^{-8}\) by
taking sequence errors into account. In line with the advice given in
previous sections, we implemented these mutation and recombination rates
for the \cite{MacLeod2013} demographic model, even though we have implemented
more recent estimates in the basic genome model. When a simulation with this
demographic model is requested, stdpopsim uses the estimates from the
demographic model rather than the estimates from the basic genome model.
Of note, this demographic model simulates genomes that represent variation
in the Holstein breed around 1980. Since then, intense selective breeding has
further reduced the effective population size, which we can simulate with
downstream breeding simulations \citep[e.g.,][]{MacLeod2014, Gaynor2020, Obsteter2021}. 

\hypertarget{conclusion}{%
\section*{Conclusion}\label{conclusion}}

As our ability to sequence genomes continues to advance, the need for
population genomic simulation of new model and non-model organism genomes is
becoming acute. So too is the concomitant need for an expandable framework
for implementing such simulations for species of interest, and
the resources for understanding when and how to do so.

In this manuscript we present the basic considerations for implementing
population genomic simulations, agnostic to simulation program. We
describe the steps of determining if a species-specific population
genomic simulation is appropriate for the species and question, what
data is necessary and why, special considerations for finding or using
that data, how to proceed when some of that data is not available,
and why we encourage everyone implementing simulations to have their
parameter choices and implementation reviewed by at least one other
researcher.

We also show how these can be integrated into the stdpopsim catalog, a
resource that is uniquely poised to fill this gap as it provides easy
access to simulations incorporating species-specific information,
easy inclusion of new species genomes, and community-maintained accuracy
and correctness (DRS: not obvious what the difference between accuracy
and correctness is from the context here.).
We additionally briefly describe how the quality control 
process for species inclusion works. Currently, large-scale efforts such as the Earth Biogenome
and its affiliated project networks are generating tens of thousands of genome
assemblies. Each of these assemblies, with some prior knowledge of mutation and
recombination rates, will become a candidate for inclusion into the
stdpopsim catalog following the steps we have outlined above. As
annotation of genome assemblies improves over time those too can easily
be added to the stdpopsim catalog.

Moreover, one of the goals of stdpopsim is to leverage stdpopsim itself
as a springboard for education and inclusion of new communities into
computational biology and software development. We are keen to use
outreach, for instance in the form of workshops and hackathons, as a way
to democratize development of population genetic simulation as well as
grow the stdpopsim catalog and library generally. By enabling
researchers of non-model species with simulation platforms that
traditionally have been quite narrowly focused with respect to organism,
we hope to raise the quality of research across a large number of
systems, while simultanousely expanding the community of software
developers at work in the population and evolutionary genetics world.
Our experience with such outreach over the past two years is that people
are indeed keen to put in the time and effort to include their favority
study species, but that simple, clear guidance is vital. Our
intention with this paper is in part to provide another learning
modality to meet that need. (DRS: these last two sentences very
clearly state one of the major goals of the paper. Maybe they should be
in the intro?)



% ILAN: moved this here from first results section. We might want to expand as a discussion point.
%
%However the advantage working together to validate population genomic
%model implementation is not restricted to species that are added to
%the stdpopsim catalog. Researchers implementing population genomic
%simulations are encouraged to have their code and parameter choices
%checked by at least one other person before using it to create
%simulations. In addition, basic genomic features of the simulation
%results (e.g. the site frequency spectrum, the extent of linkage disequilibrium, etc.)
%can be compared to the reference genome and/or known population
%genetic characteristics. In this too it is useful to have the input
%of another researcher with fresh eyes.


% ILAN: moved this here from first results section. We might want to expand as a discussion point.
%
%Finally, what about \emph{whole genome} simulations? Chromosomes
%segregate independently, so between-chromosome correlations are generally close
%to zero. But they can be occur in fairly extreme situations, such as intense
%directional or stabilising selection on multiple loci across chromosomes
%\citep{Bulmer1971, Lara2022}. However, this situation can be simulated in
%follow-up forward-in-time simulations \citep{Haller2018, Gaynor2020}. For
%this reason, we tend to simulate chromosomes independently, and few
%simulators have mechanisms to simulate
%multiple chromosomes simultaneously. 
% DRS: SLiM kind of lets you do this...
% ILAN: I decided to remove this for now.

% ILAN: moved this here from second results section. We might want to expand as a discussion point.
%
% However not all species models that were started
% at the hackathon were added, as we learned that there is a disconnect
% between what species community members wish to simulate, and those
% species that have sufficient resources for a realistic simulation.
%
% When we set out to cast a wide net and add a wide variety of species to
% the catalog, we quickly ran into species that people were enthusiastic
% to add, but lacked many (or most) of the parameters estimates discussed above. The
% utility of stdpopsim is to make complex, multifaceted population genomic models
% easily available for simulation; such data includes genetic maps, annotations, and/or
% demographic models. We have not yet encountered a species with
% widely-used demographic models but no chromosome-level assembly, so the
% main issue in practice seems to be around chromosome-level assemblies
% and around matching genome parameters to demographic models.
%
% However, there is no clear line for what level of assembly quality is
% required to be ``useful'' - the most telling indication is whether there is
% a community of users eager to use it.
%

\hypertarget{acknowledgements}{%
\section*{Acknowledgements}\label{acknowledgements}}

TODO Workshop and hackhaton attendes?

\hypertarget{funding}{%
\section*{Funding}\label{funding}}

TODO: should we order these alphabetically or in the same order as authors?

M. Elise Lauterbur was supported by an NSF Postdoctoral Research Fellowship \#2010884.

Gregor Gorjanc was supported by the University of Edinburgh and BBSRC grant to The Roslin Institute (BBS/E/D/30002275).

Andrew D. Kern and Peter L. Ralph were supported by NIH award R01HG010774.

\bibliography{references}
\end{document}
